\documentclass[a4paper]{article}
\usepackage{amsmath}
\usepackage{amsthm}
\usepackage{amssymb}
\usepackage{physics}
\usepackage{graphicx}
\newtheorem{thm}{Theorem}
\newtheorem{lem}[thm]{Lemma}
\theoremstyle{definition}
\newtheorem{definition}{Definition}[section]
\usepackage{bbold}
\numberwithin{thm}{section}
\usepackage{hyperref}
\usepackage[usenames,dvipsnames]{xcolor}
\hypersetup{
	colorlinks=true,
	linkcolor=blue,
	filecolor=magenta,
	urlcolor=cyan,
}
\urlstyle{same}
\usepackage[sort&compress,numbers]{natbib}
\bibliographystyle{naturemag}
\usepackage{doi}
\newcommand{\todo}[1]{\textcolor{red}{TODO: #1}}
\numberwithin{equation}{section}
\title{Document}
\author{Shi Feng}
\date{}
\begin{document}
\maketitle
\section{Entanglement Entropy}
In this section we describe in detail how Entanglement-related properties are to be calculated. First of all let's see how joint operators act on wavefunctions. Let $\ket{\Psi}$ be an arbitrary pure state from ED, which can be factorized into a bipartite wavefunction:
\begin{equation}
	|\psi \rangle  =  \sum_{l,r}\omega_{lr}|\psi_l \rangle  |\psi_r \rangle \equiv \hat{\Omega}
\end{equation}
where we defined $\hat{\Omega}$ to be the matrix written under the bipartite basis. The right reduced density matrix (RDM) is then obtained by tracing out the left dofs:
\begin{equation}
	\begin{split}
		\rho_r &= Tr_l [|\psi \rangle \langle \psi |]  =  \sum_{l''} \sum_{l,r}\sum_{l',r'}\omega_{lr}\omega_{r'l'}^*   \langle \psi_{l''}|\psi_l \rangle  |\psi_r \rangle \langle \psi_r' |\langle \psi_l' |\psi_{l''} \rangle\\
		       &=\sum_{l''} \sum_{l,r}\sum_{r',l'}\omega_{lr}\omega_{l'r'}^*   \delta_{l,l''}  |\psi_r \rangle \langle \psi_r' |\delta_{l',l''}\\
		       &=\sum_{lr,r'}\omega_{lr}\omega_{r'l}^* |\psi_r \rangle \langle \psi_r' |
	\end{split}	
\end{equation}
we can rewrite this as:
\begin{equation}
	\rho_r &= \sum_{lr,r'}\omega_{r'l}^*\omega_{lr} |\psi_r \rangle \langle \psi_r' |\\
	       &= \sum_{r,r'} \left(\sum_l\omega_{r'l}^*\omega_{lr}\right) |\psi_r \rangle \langle \psi_r' |
\end{equation}
that is, in the basis of $\ket{\psi_r}\bra{\psi_r'}$ the elements of RDM are 
\begin{equation}
	\rho_{r',r} = \sum_{l}\omega_{r'l}^*\omega_{lr} =  \hat{\Omega}^\dagger \hat{\Omega}
\end{equation}
In other words, if we can construct the bipartite wavefunction explicitly as a matrix form, the RDM is simply a matrix product of it with its complex conjugation. 






























%\nocite{*}
%\printbibliography
%\bibliography{references.bib}
\end{document}
